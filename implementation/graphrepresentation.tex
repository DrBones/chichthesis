The application of the RGA requires a block tridiagonal \hamil{}. Layered systems naturally lead to a block tridiagonal \hamil{} but due to the serial construction of the RGA the choice of systems is limited to this geometry. For the computation of each \gfnc{} it traverses the blocks of the $\mat{A}$ matrix first from start to end and then vice versa. It can not take any detours or traverse each element more than once per path. Because of this restriction, given a device similar to the one shown in \cref{fig:noncollinear} one has to assign all remaining nodes to a large last layer once the angled output has been reached. This results in a very large last block. As can be seen in \cref{sec:rgacomplexity} large blocks in the \hamil{} are prohibitively expensive and would effectively render the algorithm useless.
\begin{figure}[!ht]
\centering
\begin{tikzpicture}
\node at (0,0)[above] {\includegraphics[scale=0.65]{images/noncollineardevice}};
\draw[style={latex-latex,line width=2pt}] (-6,1) node[right]{$y$} -- (-6,0) node [below left]{$z$} -- (-5,0) node[below]{$x$};
\draw (-6,0) circle (4pt) [fill=black];
\node at (2,3.5){$a$};
\end{tikzpicture}
\caption{Non-collinear device. Dots denote lattice points and slices, if treated as a layered structure, are indicated by alternating white and gray colored background.}
\label{fig:noncollinear}
\end{figure}
In essence, one needs a \hamil{} with a particular block tri-diagonal structure that has a small bandwidth, i.e. small blocks. This can in principle be achieved by a process called matrix reordering. Because the diagonal elements represent the on-site energy the reordering should only exchange one diagonal element with another. In order to do so, the same permutation is applied to the rows and columns. These \emph{symmetric permutations} can be realized by a permutation martix $P$ as follows \cite{saad2003iterative}:
\begin{align}
\mat{H}' = \mat{P} \mat{H} \mat{P}^{T}\ .
\label{eqn:permutation}
\end{align}
In order to find such a permutation matrix enabling the use of the RGA one needs an algorithm adhering to the requirements of the RGA.
A potent formulation of matrix reordering can be achieved in the form of graph theory. The tight-binding \hamil{} and therefore the matrix $\mat{A}$ to invert is sparse as well as Hermitian. Thus, one can use a natural one-to-one correspondence between sparse matrices and graphs.
\subsection{Graph Representation}
A graph is a way to represent binary relations between objects of a set. A common definition of a graph is given in \cref{dfn:graph}.
\begin{dfn}\label{dfn:graph}
A \emph{graph} \textpzc{G} consists of an ordered pair $\mathpzc{G} :=(\mathpzc{N},\mathpzc{E})$, where \textpzc{N} is a non-empty set of \emph{nodes} \textpzc{n} and \textpzc{E} a set of ordered pairs of nodes $(\mathpzc{n}_1,\mathpzc{n}_2) \in \mathpzc{N} \times \mathpzc{N}$ called \emph{edges}.
A graph is called \emph{directed} if the edges observe a direction of incidence  and it is called \emph{undirected} if for any two $(\mathpzc{n}_1,\mathpzc{n}_2) \in \mathpzc{E}$ also $(\mathpzc{n}_2,\mathpzc{n}_1) \in \mathpzc{E}$.
Two nodes $\mathpzc{n}_i$ and $\mathpzc{n}_j$ are \emph{adjacent} if $(\mathpzc{n}_i,\mathpzc{n}_j) \in \mathpzc{E}$. \cite{graham1995handbook}
\end{dfn}
The \emph{adjacency graph} of a sparse $n \times n$ matrix $\mat{A}$ is a graph $\mathpzc{G}$ whose $n$ nodes represent the $n$ unknowns of the equation system the matrix is based on and the edges represent the dependence on the unknowns. For each non-zero matrix entry $\mat{A}_{i,j}\neq 0$ there exists an edge $(\mathpzc{i},\mathpzc{j}) \in \mathpzc{E}$. The matrix corresponding to an undirected graph is the \emph{adjacency matrix}. Because the \hamil{} is often Hermitian, the non-zero matrix elements are usually symmetric. Hence an undirected adjacency graph represents the matrix \emph{structure}. It is important to note that by default the graph does not hold any information about the value of the matrix elements but only their relation. However it is a simple matter of storing the values of the matrix elements as a \emph{weight} of the edges.\par
A graph may be pictured as a collection of circles representing the nodes and the connecting lines representing the edges. Incidentally, for tight-binding models each circle corresponds to a site and the lines depict the hopping options. If the graph is weighted the thickness of a line may represent the hopping \emph{possibility}. An example for a small number of grid points can be seen in \cref{fig:adjacencygraph}.
\begin{figure}[!h]
  \centering
  \subfloat[]{\label{fig:discretizeddevice}\includegraphics[scale=1]{images/noncollineardevicelevelstructure}}\hspace{2.35em}
  \subfloat[]{\label{fig:adjacencygraph}\includegraphics[scale=0.4]{images/7x6l_graph}}\hspace{2.35em}
  \subfloat[]{\label{fig:hamilmatrix}
  \begin{tikzpicture}
  \def\a{0.51}
  \def\b{1.28}
  \def\c{0.88}
  \node at (0,-0.04){\includegraphics[trim= 0 0 0 2pt,clip=true,scale=1]{images/7x6lhamil}};
  \draw[dashed,thick] (-1.6,1.2) -- ++(\a,0)--++(0,-\a)--++(-\a,0) -- cycle;
  \draw[dashed,thick] ($(-1.6+\a,1.2)$) -- ++(\b,0)--++(0,-\a)--++(-\b,0) -- cycle;
  \draw[dashed,thick] (-1.099,0.69) -- ++(\b,0)--++(0,-\b)--++(-\b,0) -- cycle;
  \draw[dashed,thick] ($(-1.099+\b,0.69)$) -- ++(\c,0)--++(0,-\b)--++(-\c,0) -- cycle;
  \draw[dashed,thick] ($(-1.099-\a,0.69)$) -- ++(\a,0)--++(0,-\b)--++(-\a,0) -- cycle;
  \draw[dashed,thick] (0.19,-0.59) -- ++(\c,0)--++(0,-\c)--++(-\c,0) -- cycle;
  \draw[dashed,thick] ($(0.19+\c,-0.59)$) -- ++(\a,0)--++(0,-\c)--++(-\a,0) -- cycle;
  \draw[dashed,thick] ($(0.19-\b,-0.59)$) -- ++(\b,0)--++(0,-\c)--++(-\b,0) -- cycle;
  \draw[dashed,thick] (1.08,-1.49) -- ++(\a,0)--++(0,-\a)--++(-\a,0) -- cycle;
  \draw[dashed,thick] ($(1.08-\c,-1.49)$) -- ++(\c,0)--++(0,-\a)--++(-\c,0) -- cycle;
  \end{tikzpicture}}
  \begin{tikzpicture}[overlay]
  \draw[-latex,line width=2pt] (-12.5,-1)--++(1,0);
  \draw[-latex,line width=2pt] (-6.25,-1)--++(1,0);
  \end{tikzpicture}
  \caption{(a) Schematic of a simple discretized non-collinear device. The dots denote lattice points. The levels are indicated by alternating white and gray background. (b) The relations between nodes are stored in the edges of the depicted graph. Nodes are represented as red circles. (c) \hamil{} of the discretized nanowire obtained by matrix reordering. The blocks are highlighted by dashed lines. The arrows indicate the matrix reordering procedure.}
  \label{fig:matrixtograph}
\end{figure}
\subsection{Matrix Reordering as a Graph Partitioning Problem}
The process of matrix reordering with symmetric permutations corresponds to renaming of the nodes of a graph without altering the edges. In order to find the permutation matrix $\mat{P}$ one has to sort the nodes of the graph.\par
There exist numerous algorithms that can reduce the bandwidth of the adjacency matrix like the well known reverse \textsc{Cuthill-McKee} algorithm but the application of the RGA imposes an additional constraint. It requires that the last block of the matrix (the end nodes of the graph) has to be the endpoint of the walk of the algorithm, i.e. the sites adjacent to a contact have to be in the upper-left and lower-right corners of the matrix. That means the nodes representing these sites may not be renamed. A \emph{partition} of the graph yields a so-called \emph{level structure} \cite{gibbs.Siam.13.236}.
\begin{dfn}
A level structure $\mathpzc{L}(\mathpzc{G}) = (\mathpzc{l}_0,\mathpzc{l}_1,\mathpzc{l}_2,\dotsc,\mathpzc{l}_N)$, of a graph \textpzc{G} is a partition of the set of nodes \textpzc{V} into $N$ \emph{levels} $\mathpzc{l}_i$ such that
\begin{enumerate}
\item all nodes adjacent to nodes in level $\mathpzc{l}_1$ are in either level $\mathpzc{l}_1$ or $\mathpzc{l}_2$
\item all nodes adjacent to nodes in level $\mathpzc{l}_N$ are in either level $\mathpzc{l}_{N}$ or $\mathpzc{l}_{N-1}$
\item for $1 < i < N$, all nodes adjacent to nodes in level $\mathpzc{l}_{i}$ are either in level $\mathpzc{l}_{i-1}$, $\mathpzc{l}_{i}$ or $\mathpzc{l}_{i+1}$
\end{enumerate}
\end{dfn}
For a simple collinear geometry a level is usually identical to a slice. For more complex geometries a level and a slice may differ. A possible choice to fulfill these requirements is the \textsc{Gibbs-Poole-Stockmeyer} (GPS) algorithm \cite{gibbs.Siam.13.236}. Two conditions are of crucial importance for the reordering in order to achieve high RGA performance. As the complexity of the RGA is of order $\mathcal{O}(N^3)$(\cref{sec:rgacomplexity}) large blocks have to be avoided. In addition, a balanced level structure has to be sought, i.e. ideally,  all the levels have the same number of nodes.
An algorithm tailored to the demands of high RGA performance has been introduced by \textsc{Wimmer}. As it outperforms the GPS algorithm and offers greater versatility \cite{Wimmer2009JComPhys} it is applied to the adjacency graph of the \hamil{} prior to the execution of the RGA.
\subsection{RGA Tailored Block-Tridiagonalization Algorithm}
The process of graph partitioning is known to be NP-Complete \cite{GareyTCS.1.237}. Hence, one cannot compute the optimal level structure from first principles but has to employ heuristics. The algorithm used by \textsc{Wimmer} incorporates a global and a local approach. To find the maximum number of levels for a given geometry a \emph{Breadth-First-Search} (BFS) as described in \cref{alg:bfs} is performed as it yields a level structure by construction as well as the maximum number of levels $N_0$.
\begin{algo} \label{alg:bfs}
\textit{Breadth-First-Search Algorithm}\\
\begin{tabularx}{\textwidth}{l}
  \addlinespace \cmidrule(r{7.6cm}){1-1}
\begin{minipage}{\textwidth}
    \vskip 4pt
    \begin{enumerate}[1]
   \item \textbf{Start} with $i=1$ then $\mathpzc{l}_i = \mathpzc{l}_1$ because start and end nodes are fixed
   \item If any node $\mathpzc{n} \in \mathpzc{l}_i$ is adjacent to a $\mathpzc{n} \in \mathpzc{l}_N$ distribute all nodes not in $\mathpzc{l}_i$ to $\mathpzc{l}_N$ and \textbf{end}
   \item Assign all nodes not in levels $\mathpzc{l}_0 - \mathpzc{l}_i$ to level $\mathpzc{l}_{i+1}$
   \item \textbf{Continue} at 2 with $i=i+1$
   \end{enumerate}
   \vskip 4pt
 \end{minipage}
\\
 \bottomrule 
\end{tabularx}
\end{algo}
With the maximum number of levels $N_i$ in $\mathpzc{l}_i$ known the graph is cut into sections $\mathpzc{l}_{i_x}$ with approximately equal number of levels to ensure a balanced level structure, i.e.
\begin{align}\label{eqn:balancecriterion}
\abs{\mathpzc{l}_{i_1}} \approx \frac{N_{i_1}}{N_i}\abs{\mathpzc{l}_{i_1}} \qquad\text{and}\qquad \abs{\mathpzc{l}_{i_2}} \approx \frac{N_{i_2}}{N_i}\abs{\mathpzc{l}_{i_1}}\ .
\end{align}
With $\abs{\mathpzc{l}_i}$ denoting the number of nodes in that level.  All the nodes in each of those sections are distributed into levels by a BFS and optional further optimization starting from both ends where the graph has been cut. This section is in turn cut into ideally equal number of levels and the process is recursively repeated until only one level remains. The ordered collection of these final levels of all recursive paths constitutes the sought level structure.\par
If the nodes have been named from 1 to $k$ with $k$ being the size of the square \hamil{} the matrix reordering is achieved by simply rearranging the rows and columns symmetrically according to the sequence of the nodes in the level structure. The ordering of the nodes within one level is of minor importance. 
A modification of this algorithm omitting detailed optimizations as used in this work is presented in \cref{alg:wimmeralgorithm}.\par
% \noindent\begin{minipage}{\textwidth}
\begin{algo}\label{alg:wimmeralgorithm} 
\textit{Block-Tridiagonalization Algorithm}\\
\begin{tabularx}{\textwidth}{l}
 \addlinespace\cmidrule(r{7cm}){1-1}
\begin{minipage}{\textwidth}
    \vskip 4pt
    \begin{enumerate}[1]
 \item Generate graph \textpzc{G} corresponding to the \hamil{} and the starting level $\mathpzc{l}_0$ and ending  level $\mathpzc{l}_N$.
   \item Use BFS to determine maximum number of levels $N$.
   \item Bisect remainder of graph: $\mathpzc{G}_1 = \mathpzc{G} \setminus (\mathpzc{l}_0 \cup \mathpzc{l}_N)$ containing $N_i = N-2$ levels. With left-adjacent level $\mathpzc{l}_l = \mathpzc{l}_0$ and right-adjacent level $\mathpzc{l}_r = \mathpzc{l}_N$.\label{alg:wimmeralgorithm3}
   \item[] \begin{enumerate}[a]
   \item  \textbf{Stop}, if $N_i$ = 1.\label{alg:wimmeralgorithm3a}
   \item Do a BFS starting from level $\mathpzc{l}_l$ up to level $N_{i_1} = \text{Floor}(N_i/2)$ and a BFS starting from level $\mathpzc{l}_r$ up to level $N_{i_2}= N_i - \text{Floor}(N_i/2)$. The nodes found are assigned to $\mathpzc{l}_{i_1}$ and $\mathpzc{l}_{i_2}$ respectively and marked as visited.
   \item Distribute remaining nodes in level. Continue BFSs from previous step from both sides and assign nodes according to the balance criterion (\ref{eqn:balancecriterion}).
   \item Recursively apply step (\ref{alg:wimmeralgorithm3}) for levels $\mathpzc{l}_{i_1}$ and $\mathpzc{l}_{i_2}$ until in (\ref{alg:wimmeralgorithm3a}) final level is reached.
   \end{enumerate}
   \item Collect all levels from (\ref{alg:wimmeralgorithm3}) into level structure \textpzc{L} and reorder matrix rows and columns according to the sequence of nodes found  in \textpzc{L}
   \end{enumerate}
   \vskip 4pt
 \end{minipage}
\\
 \bottomrule 
\end{tabularx}
\end{algo}
% \end{minipage}\vskip 1em\par
The steps involved in the matrix reordering are sketched in \cref{fig:discretizeddevice,fig:adjacencygraph,fig:hamilmatrix}.
By reordering the matrix one can use the unaltered RGA even for more complex geometries but also gains the ability to this conventional two-terminal algorithm for multiple terminals.
\FloatBarrier
