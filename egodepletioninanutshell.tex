%!TEX root = main.tex
\onehalfspacing
Should I get up, or should I stay in bed for another five minutes? Should I wear my blue or my black jeans today? “\emph{Should I stay or should I go?}” (The Clash, 1982). The average adult human makes approximately 20.000 decisions a day \parencite{toennemann2008}. Most of them are everyday choices, which are made almost subconsciously. However, some of those decisions have greater impact on someone’s life than others. Therefore, difficult choices often ask for a balanced decision process or even include taking a calculated risk.\par
Risk behaviour in decision-making is a potentially interesting field of research for enter-prises \parencite{baumeister2002yielding} when it comes to online purchase. How does online shopping modi-fy consumer behaviour and how can adjustment to the needs of customers look like? Alt-hough, it appears that consumers demand more choice in every product (e.g., colour, shape, size, functionality, endurance, etc.) research has shown that with a significant increase of choice and subsequently decision-making, consumers are inclined to reject products they nor-mally would purchase \parencite{greifeneder2010less,levav2010order}. Therefore, the question is why do people start rejecting products they would usually buy and why do they buy the product if less decision-making is involved?\par
The concept of decision-making in the context of consumer choice becomes even more complex when distinguishing between ‘risk behaviour’ on the one hand and ‘social-risk behaviour’ on the other. Social risk addresses a new issue in the debate of ego depletion and consumer behaviour. The present thesis deals with the effects of ego depletion on consumer decisions and risk-taking behaviour. Based on the research of Unger and Stahlberg (2011), who found support for the assumption that people become risk averse under ego-depletion when it comes to financial risk, we assume that ego-depletion also leads to risk-averse decision-making when it comes to social risk.\par
In order to provide a brief introduction into the theoretical foundations of this thesis, chapter two is dedicated to closer examine the research on ego depletion, self-regulation, deci-sion-making, as well as risk and social risk. The second chapter will also include a precise def-inition of the term ‘social risk’ (as used in this thesis) versus ‘risk’ in order to clearly dissociate both concepts. \par
Chapter three is dedicated to the empirical research. It includes a description of the pre test that will be administered in order to determine and finally select products that involve a low and a high social risk, respectively, to be used in the final survey. Furthermore, a similar procedure will be used in order to design possible imaginary scenarios that involve the same levels of social risk. The aim is to design a questionnaire in which participants will have to de-cide between two products or two scenarios, which either involve a relatively low level of social risk or a relatively high level of social risk. A distinction is made between social risk in decision-making \parencite{danziger2011extraneous} (i.e., scenarios) and social risk in consumer decisions (i.e. products) with the intention to distinguish between potential implications of social risk.\par
With a pre-approved selection of products and scenarios containing the two different so-cial risk conditions, chapter three will also present the design of the questionnaire itself. The design process will furthermore include the selection of a depletion task to controllingly ma-nipulate participants’ ability for self-regulation and a self-evaluation measure in order to assess participants’ consumer behaviour and the perception on their ability (or lack of ability) of self-regulation. This information might provide us data regarding the subject’s self-perception.\par
Chapter four deals with the empirical analysis of the results. It will contain several tests in order to profoundly check upon manipulation regarding the experimental group (depletion condition) and the control group (non-depletion condition) to rule out any interference unre-lated to ego-depletion. Furthermore, we intend to diminish the data regarding the self-evaluation part. Conducting a factor analysis will help to decide whether the items can be merged into one latent variable. This approach might simplify the process of testing group dif-ferences with respect to the means. To strengthen the information value of the data we decid-ed to conduct a second test additional to the product consumer choice. This test aims to test decision-making under social risk itself  (not under the condition of consumption).\par
The fifth and last chapter of the thesis will contain a critical discussion of the instru-ments and the obtained data and will also provide an outlook on future research.\par