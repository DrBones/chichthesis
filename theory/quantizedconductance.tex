In transport measurements of mesoscopic systems at low temperatures a quantized conduction has been observed \cite{PhysRevLett.45.494}. Thus, the conductance of mesoscopic systems exhibits peculiarities with no classical analogon.
\subsection{\clanbform{}}\label{sec:landauerbuettiker}
In an ideal lattice the electrons would move like in vacuum (empty lattice) but with an effective mass. Impurities or phonons introduce scattering to the system altering the momentum of moving electrons through collisions.
If the size of the electronic device shrinks below the mean free path of an electron, i.e. the distance an electron has to travel until its initial momentum is destroyed, the transport becomes ballistic\,\cite{datta1989quantum}. 
\begin{align}
	\text{length of device} \lesssim \lambda_{\text{mfp}}\quad \Rightarrow \quad\text{ballistic regime}
	\label{eqn:meanfreepath}
\end{align}
Due to the lack of scattering one would suspect zero resistivity. Nonetheless a finite resistivity which is quantized as a function of the width of the conductor was found \cite{PhysRevLett.60.848}.
A popular approach to the effects of nano-scale devices initialized by \textsc{Landauer} \cite{PhilMag.21.863} and extended by \textsc{B\"uttiker} is the so-called \lanbform{} \cite{PhysRevB.31.6207}.
\begin{figure}[h]
\centering
\begin{tikzpicture}
      % \draw[step=.5cm,gray,very thin] (-4,0) grid (4,4);
      \node at (0,0)[above] {\includegraphics[width=.5\textwidth]{images/landauer}};
      \node at (-4,4)[below] {$T_1,\mu_1,f_1$};
      \node at (3,4)[below] {$T_2,\mu_2,f_2$};
      \node at (4,0.5)[above] {$T_3,\mu_3,f_3$};
      \node at (-0.1,2)[align=center] {Scattering\\ region};
\end{tikzpicture}
\caption{Schematic of device in \lanbform{} with three reservoirs and leads. The blue area is the active scattering region. Each reservoir has distinct Temperature $T_i$, chemical potential $\mu_i$ and therefore \textsc{Fermi} distribution $f_i$.}
\label{fig:lanbform}
\end{figure}
The \textsc{Landauer-B\"uttiker} model is in essence quite general but all devices are composed of a scattering region and one or more leads. The device of interest is theoretically divided into a scattering area, henceforth called conductor, and multiple leads connecting the conductor to macroscopic reservoirs. All reservoirs and their adjacent leads are assumed to be in thermal equilibrium. Each lead has a distinct \textsc{Fermi} distribution $f_R$ of temperature $T_R$ and chemical potential $\mu_R$, see \cref{fig:lanbform}.
The \lanbform{} relates the current flow between leads to the chemical potential of the reservoirs. In hindsight of the inclusion of spin, the expressions are given per spin state of the electron. Otherwise, a factor of 2 for spin degeneracy would have to be included in the calculations. The current in lead $p$ \cite{PhysRevLett.68.2512}
\begin{align}
I_p=\frac{-e}{h} \sum_q \int T_{pq}(E) [f_p(E) - f_q(E)] \text{ d}E.
\label{eqn:current}
\end{align}
results from the leads $q$ with their respective \textsc{Fermi} distributions $f_i(E)$ via the tranmission coefficients $T_{pq}(E)$. The current includes the electron charge $e$ and the \textsc{Planck} constant $\hbar$. The \lanbform{} relates the conductance of the device with the possibility of an electron passing through it. The transmission coefficients $T_{pq}(E)$ state the probability of an electron traveling from lead $q$ to lead $p$. 
In the linear response regime with low temperatures the expression can be linearized 
\begin{align}
I_p=\frac{-e}{h} \sum_q T_{pq}(E_F) [\mu_p - \mu_q].
\label{eqn:currentlin}
\end{align}
In the low temperature regime the \textsc{Fermi} distributions can be replaced by their respective \emph{chemical potentials} $\mu_i$. The possibility that the energy $E$ can be replaced by the \textsc{Fermi} energy $E_F$ shows that the conductance is effectively a \emph{\textsc{Fermi} surface} property depending only on states near the \textsc{Fermi} energy.
\subsection{Conductance from Transmission}\label{sec:conductancefromtransmission}
Assuming that the conductance can be expressed in terms of current and chemical potential results in a relation between the conductance from lead $q$ to lead $p$ and the transmission coefficients 
\begin{align}
G_{pq}=\frac{\abs{I\cdot e}}{(\mu_p - \mu_q)}=\frac{e^2}{h} T_{pq}(E_F).
\label{eqn:conductance}
\end{align}
The conductance can be calculated by quantum mechanical methods. The transmission coefficients $T_{pq}(E)$ are directly related to the \emph{transmission probability amplitudes} $t^{pq}_{ll'}$ leading to
\begin{align}
G_{pq}=\frac{e^2}{h} T_{pq} =\frac{e^2}{h} \sum_{ll'} \abs{t^{pq}_{ll'}}^2.
\label{eqn:transcoeff}
\end{align}
The transmission probability amplitudes $\abs{t^{pq}_{ll'}}$ describe the electron flux amplitude for an electron traveling from channel $l'$ in lead $q$ to channel $l$ in lead $p$. This definition only holds for leads that are longitudinally translational invariant. The transmission coefficients are matrix elements of a so-called \emph{scattering wavefunction} which governs the electron flow from and to the leads \cite{Datta1997}. There exists an alternative method to direct quantum mechanical evaluation of the scattering wave function to obtain the transmission coefficients as will be shown in \cref{sec:observables}.
