%!TEX root = main.tex
\onehalfspacing
The aim of the present experiment was to elicit the effects of ego depletion on consumer decisions regarding social risk-taking behaviour. In addition, we tried to investigate how ego depletion moderates social risk-taking behaviour regarding judicial cases embedded in a social context (non-consumption terms) in order to support our data. \par
Results of the experiment did not support our assumption that ego depletion leads to a ‘social risk averse decision-making’ when it comes to consumer behaviour. Furthermore, our results indicate that ego depletion had no effect on the social risk related judicial decisions either. Even though some of the results, both of the consumer choice and the judges’ dilemma, revealed a slight tendency in favour of our assumption, they did not reach significant levels in the empirical analysis. \par
To our astonishment and absolutely contrary to our expectations, two results we obtained were pointing into the opposite direction. In reference to the product ‘sofa’ (see Section \ref{sec:pearson_chi_products}) the experimental group leaned towards the high ‘social risk’ and the control group to-wards the low ‘social risk’ alternative. Among the product choice tasks, this result was unique, however, akin to the result of judges’ case 3 (see Section \ref{sec:pearson_chi_judges}). Therefore, this phenomenon did not just occur related to purchase decisions, but also in cases of social risk completely un-related to consumer decisions. Even though the depletion measure task (see Section \ref{sec:depletion_measure_selfevaluation}) did indicate a higher depletion rate in the experimental group compared to the control group, the self-evaluation part of the questionnaire regarding the rate of ‘exhaustion’ did not comply with this result. Interestingly, the self-evaluation with regard to ‘difficulty’ did comply with the result of the depletion measure task. Therefore, it can be assumed that the subjects in the experimental group were actually in a state of depletion.\par
Possible interference from other variables such as age, gender, time (Stroop test), and dif-ferences in self-evaluation regarding this sample was thoroughly tested and could be ruled out to account for the results. \par
\section{Critical Examination of the Experiment}
The theoretical framework was based on Baumeister’s theory of ego depletion, which pos-tulates that willpower is a scarce source that drains from the same origin as energy used in decision-making processes and self-regulation. This assumption was corroborated by a consid-erable number of researchers who supported and extended this theory (compare Chapter 2.1 and 2.2). Unger and Stahlberg examined the effects of ego depletion on risk behaviour \citep{unger2011ego}. Although they were able to confirm that depleted subjects behave risk-averse when it comes to decision-making, the effect was only verified with regard to financial risk. They used investment options (participants had to act as managers who had to decide in which country to invest) in order to design low and high risk scenarios. This design was not adaptable for our purposes, as financial risk does not fall under the definition of social risk (see chapter 2.3). Judging by Unger and Stahlberg’s results, however, it appears as if financial risk had a stronger impact on decision-making than social risk. Obviously, the threat of losing big amounts of money is a considerably higher (if not existential) risk than buying a sofa or a pair of shoes, and risk-averse behaviour might therefore be easier to trigger.\par
Even though researchers have dedicated themselves to the concept of social risk in the past (compare Chapter 2.3), it appears as if a consistent terminology has not yet been successfully established. Moreover, the current debate on social risk still lacks a standard evaluation meth-od. However, the results of this study call into question if the instruments used in this study were apt to measure the effects of ego depletion. Precisely, the ambiguity and vagueness of our results make it difficult to come to a general conclusion, positive or negative. The result neither confirmed our hypotheses nor could it be distinctively rejected.  Due to these results the terminology as well as structure of the questionnaire need to be scrutinized. Consequently, we will closer examine the pre-test, the depletion task, the online questionnaire, and our find-ings in the following section.\par
The pre-test was primarily designed to examine the different ‘loads’ of social risk among a selection of products that differed in only one attribute. It appeared to us, that our intention to emphasize the scale visually in the questionnaire as a grey shadowed triangle behind the scale created an optical illusion to the eye. It seemed, that the boxes (to mark the answer) appeared to become smaller from left to right (low risk to high risk), even though they were all the same size, implying that there was a preferable answer. This, however, can be explained with the grey triangle in the background of the boxes. It is possible that this trick of the eye could have influenced the response behaviour of the participants. \par

\begin{figure}[H]
		\begin{tikzpicture}[text depth=3pt]
		\tikzstyle{every node}=[style={font=\footnotesize\vphantom{Ag}}]
		\node at (0,0) {\includegraphics[trim= 2.1cm 0cm 2cm 0cm,clip=true,width=290pt]{images/pretest_scale}};
		\node at (-2.7,2.28) {\textnormal{(depletion measure)}};
		% trim = l b r t
		% \draw (-3,-3) to[grid with coordinates] (3.5,3);
		\end{tikzpicture}
		\caption{Pre test scale (own design)}\label{fig:pretest_scale}
\end{figure}
However, the questionnaire provided us with three different possible approaches to evalu-ate social risk with regard to product decision-making. The first approach, which was adapted in the online questionnaire, aimed for two (out of five) products with a preferably equal ‘liking’ rating and a high social risk diversity at the same time (see Chapter 3.1.1). In a second approach we assessed the aspect of social risk regarding the products in general (not ranked according to a hierarchy of five alternatives), e.g. the social risk of the product ‘shoes’ itself. In that case the ‘liking’ rate was irrelevant. Instead, we aimed for the perceived social risk intensity (one product in general). In the third approach we assessed the relevance of social risk and product design to participants’ purchase decisions. Hence, the fundamental difference between the first and the third approach is that while the former evaluated an individual level of social risk for each variant of a product (e.g. product ‘shoes’ in colour ‘red’), the latter evaluated the importance of social risk for participants’ purchase decision with regard to product in general (e.g. product ‘shoes’). We came to the conclusion that for our needs an equal ‘liking’ rate was the best way to establish a basic condition upon which to compare dif-ferent levels of social risk. In the light of our findings, however, we advise that future research should incorporate the relevance of social risk to purchase decisions, as it seems to play a sub-stantial role in the depiction of consumer behaviour under social risk.\par
The pre-test for the judges’ cases was based on a research project carried out at Ben Gurion University and Columbia University (compare subsection 3.2.3). According to the findings of this research, we originally designed eight different judicial cases, of which three were fi-nally chosen to be part of the online questionnaire, as a ‘control experiment’ for the product choices. Even though their results have shown that repeated ruling as a version of self-regulation led to a tendency deciding in favour of the status quo (leaving the perpetrator in custody, instead of granting probation), Levav and his colleagues ask to consider that “\emph{[\ldots] although we interpret our findings through the lens of mental depletion, we do not have a direct measure of the judges’ mental resources and, thus, cannot assess whether these change over time}” \citep[p.~6892]{danziger2011extraneous}. Nevertheless, in the light of Levav’s findings which sug-gest that self-regulation leads to depletion, we decided to implement different judicial scenarios in the questionnaire to assess whether our depletion task also influences choices other than those involved in purchase decisions. In this context, the social risk was implied by the potential consequences of the participants’ decision, e.g. granting probation with the risk of releasing a possible repeat offender. \par
In the scientific discourse surrounding ego depletion several self-regulation tasks were developed and conducted (e.g. crossing out the vowel ‘e’ in a text, watching a repulsive video while keeping a straight face, etc.) All of them aimed at depletion by evoking self-regulation processes. Considering the impossibility to monitor participants during an online questionnaire, in this case the Stroop test seemed the most practical choice of depletion. Moreover, “\emph{[a]dditional analyses indicated that the depletion manipulation (difficult versus easy Stroop task) had no significant effect on self-reported tiredness or negative affect. However, it produced a significant reduction in positive affect, which appeared to be a side effect of the manip-ulation rather than a part of the underlying process}“ \citep[p.~350]{pocheptsova2009deciding}.\par
The results of the depletion measure task (compare subsection 4.2.5) affirmed our ex-pectation that the Stroop test did indeed deplete the experimental group compared to the con-trol group. Although the control group showed a lower level of depletion after the Stroop test and the questionnaire, we cannot rule out the possibility that the alternate task for the control group evoked at least a minor state of depletion, which could have biased the results.\par
The fact that the questionnaire was designed to fit in an online format enabled us to monitor not just the answers, but also the time participants needed to complete the Stroop test. The advantage of conducting an online questionnaire was the option to easily reach possible subjects worldwide. However, in order to send out an online questionnaire internationally it had to be designed in English. This leads to the question if participants encountered problems comprehending the instructions and the tasks. Even though we tried to keep all instructions and questions as simple as possible, some of them might have been challenging for some par-ticipants, which could also account for biased results. Furthermore, we did not have any su-pervision over participants’ mood before and during the questionnaire. It might be essential to the results if participants conducted the questionnaire under stress or in a relaxed condition. Besides, even the time of day might make a difference. Participants’ response behaviour might easily vary after a stressful day’s work.\par
Although the advantages of an online questionnaire are undeniable, they do not com-pensate the disadvantages compared to a laboratory study. With regard to the monitoring pro-cess the environment of a laboratory would have made it easier to evaluate participants’ mood before conducting the study. Furthermore, it would have provided us with the advantage to monitor the whole working process of each participant in order to eliminate any diversions which subjects are generally exposed to when they are not in a controlled environment (e.g. telephone calls, coffee making, email notifications, etc.). Finally, we have to consider that the factor time (mean = 25 min. and 3 sec.) might have led participants to rush through the ques-tionnaire instead of considerately answering each question.\par
Taking all into consideration, a replication of the study under laboratory conditions seems advisable and promising. \par
\section{Outlook}
Even though the results of this study did not comply with our expectations, we are still convinced that further research could yield valuable results, as effects of ego depletion on decision-making under social risk remain a wide field. Nevertheless, our results also indicate that several adjustments are needed to successfully study the effects of ego-depletion on deci-sion-making behaviour under social risk and achieve convincing results.\par
First, difficulties in measuring social risk experienced during the course of this study sug-gest the necessity of creating a psychometric scale, which exclusively samples social risk, in order to establish a more solid basis for upcoming work in this field. This proposition is consistent with Weber, Blais and Betz’s (2002) findings, which supported the hypothesis that risk-taking behaviour is domain specific (e.g. financial risk, health/safety risk, social risk, etc.). In other words, risk-behaviour does not follow a ‘one size fits all’ pattern. It depends not only on the circumstances but also on the kind of risk (domains). Consequently, they designed a new risk attitude scale that also incorporates perceived risk attitudes as well as gender differences. Moreover, it appears most sensible to further explore the phenomenon of social risk to better understand the impact of ego depletion on decision-making under social risk in the field of marketing and consumer behaviour. We assume that only if social risk can be clearly distin-guished from other domains of risk, it is possible to further investigate its effects in connec-tion with ego depletion on decision-making processes.\par
Secondly, based on a scale that is able to reliably measure social risk, a follow-up study would require a revision of the choice of products that were used in our questionnaire. We propose that the products should not only be chosen by a similar ‘liking’ and ‘high social risk diversity rate’, but also by the ranking of the social risk of a product in general and by the impact that social risk has on participants’ purchase decisions.\par
Thirdly, as mentioned above, the participants, the experiment and especially the self-regulation tasks (e.g. Stroop test) should be monitored under laboratory conditions. An online questionnaire does not provide the necessary surveillance, which we assume is needed to rule out any disturbances during the execution of the experiment. Furthermore, laboratory conditions could ensure that no strong deviances of participants’ conditions (e.g. well rested in the morning, headache, etc.) occur, which might interfere with their performance on the tasks. \par
In this thesis the attempt was made to elicit the effect of ego depletion on consumer decisions and risk-taking behaviour. Whilst this study did not confirm our initial hypothesis, tendencies among participants that hint towards social risk aversion under ego depletion could be sensed. We are convinced that the findings of this study provide a possible number of im-plications for future practice and that the information thus obtained can be used to develop new research models to continue examining the effects of ego depletion on consumer deci-sions and risk-taking behaviour.\par
